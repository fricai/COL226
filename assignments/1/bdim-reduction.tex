\documentclass{scrartcl}

\usepackage{minted}
\renewcommand{\theFancyVerbLine}{\sffamily\arabic{FancyVerbLine}}

\title{
	COL226 Assignment 1\\
	Basic Decimal Integer Machine
}
\date{13 January 2022}
\author{Rishabh Dhiman}

\begin{document}

\maketitle

\section{Absolute Value}

\begin{enumerate}
\item \begin{minted}{cpp}
cin >> x;
if (x <= 0) x = -x;
cout << x;
\end{minted}

\item Get rid of constants and unary operators.

\begin{minted}{cpp}
zero = 0;
cin >> x;
if (x <= zero) x = zero - x;
cout << x;
\end{minted}

\item Write branches and loops as goto
\begin{minted}{cpp}
zero = 0;
cin >> x;
if (x > zero) goto OUT;
x = zero - x;
OUT: cout << x;
\end{minted}

\item Write as code/mem state machine.
\begin{minted}[linenos,firstnumber=0]{cpp}
mem[0] := 0;
input: mem[1];
mem[2] := mem[1] > mem[0];
if mem[2] goto code[5];
mem[1] := mem[0] - mem[1];
output: mem[1];
halt;
\end{minted}

\item Convert to BDIM with wildcards.
\begin{minted}[linenos,firstnumber=0]{text}
(16,0,_,0)
(1,_,_,1)
(12,1,0,2)
(13,2,_,5)
(7,0,1,1)
(15,1,_,_)
(0,_,_,_)
\end{minted}
\end{enumerate}

\section{Arithmetic Progression}
\[\sum_{i = 0}^{n - 1} (a + id) = na + \frac{n(n - 1)}{2} d\]
\begin{enumerate}
\item \begin{minted}{cpp}
cin >> a >> d >> n;
cout << (n * a + n * (n - 1) / 2 * d);
\end{minted}

\item Get rid of temporary values.

\begin{minted}{cpp}
cin >> a;
cin >> d;
cin >> n;
x = 1;
y = n - x;
y = n * y;
y = y * d;
x = 2;
y = y / x;
x = n * a;
x = x + y;
cout << x;
\end{minted}

\item Write as code/mem state machine.
\begin{minted}[linenos,firstnumber=0]{cpp}
input: mem[0];
input: mem[1];
input: mem[2];
mem[3] := 1;
mem[4] := mem[2] - mem[3];
mem[4] := mem[2] * mem[4];
mem[4] := mem[4] * mem[1];
mem[3] := 2;
mem[4] := mem[4] / mem[3];
mem[3] := mem[2] * mem[0];
mem[3] := mem[3] + mem[4];
output: mem[3];
halt;
\end{minted}

\item Convert to assembly-BDIM
\begin{minted}[linenos,firstnumber=0]{text}
(input,_,_,0)
(input,_,_,1)
(input,_,_,2)
(read,1,_,3)
(sub,2,3,4)
(mul,2,4,4)
(mul,4,1,4)
(read,2,_,3)
(div,4,3,4)
(mul,2,0,3)
(add,3,4,3)
(output,3,_,_)
(halt,_,_,_)
\end{minted}

\item Convert BDIM with strings to BDIM using code.
	\inputminted[linenos,firstnumber=0]{text}{bdim/ap.bdim}
\end{enumerate}

\section{Factorial}
\begin{enumerate}
\item \begin{minted}{cpp}
cin >> n;
f = 1;
i = 0;
while (i < n) {
	++i;
	f = f * i;
}
\end{minted}

\item Write branches and loops as goto
\begin{minted}{cpp}
cin >> n;
f = 1;
i = 0;
L: i = i + 1;
f = f * i;
if (i < n) goto L;
cout << f;
\end{minted}

\item Get rid of temporary values and constants.

\begin{minted}{cpp}
cin >> n;
one = 1;
f = 1;
i = 0;
L: i = i + one;
f = f * i;
b = n > i;
if b goto L;
cout << f;
\end{minted}


\item Write as code/mem state machine.
\begin{minted}[linenos,firstnumber=0]{cpp}
input: mem[0];
mem[1] := 1;
mem[2] := 1;
mem[3] := 0;
mem[3] := mem[3] + mem[1];
mem[2] := mem[2] * mem[3];
mem[4] := mem[0] > mem[3];
if mem[4] goto code[4];
output: mem[2];
halt;
\end{minted}

\item Convert to BDIM-string
\begin{minted}[linenos,firstnumber=0]{text}
(input,_,_,0)
(read,1,_,1)
(read,1,_,2)
(read,0,_,3)
(add,3,1,3)
(mul,2,3,2)
(gt,0,3,1)
(if,4,_,4)
(output,2,_,_)
(halt,_,_,_)
\end{minted}

\item Convert BDIM-string to BDIM using code.
	\inputminted[linenos,firstnumber=0]{text}{bdim/fact.bdim}
\end{enumerate}

\section{Fibonacci}
\begin{enumerate}
\item \begin{minted}{cpp}
cin >> n;
a = 0;
b = 1;
for (int i = 0; i < n; ++i) {
	a = a + b;
	c = a;
	a = b;
	b = c;
}
cout << a;
\end{minted}

\item Write branches and loops as goto
\begin{minted}{cpp}
cin >> n;
a = 0;
b = 1;
i = 1;
L: if (i > n) goto E;
a = a + b;
c = a;
a = b;
b = c;
i = i + 1;
goto L;
E: cout << a;
\end{minted}

\item Get rid of temporary values and constants.

\begin{minted}{cpp}
cin >> num;
a = 0;
b = 1;
i = 1;
one = 1;
L: cmp = i > num;
if (cmp) goto E;
a = a + b;
c = a;
a = b;
b = c;
i = i + one;
goto L;
E: cout << a;
\end{minted}


\item Write as code/mem state machine.
\begin{minted}[linenos,firstnumber=0]{cpp}
input: mem[0];
mem[1] := 0;
mem[2] := 1;
mem[3] := 1;
mem[4] := 1;
mem[5] := mem[3] > mem[0];
if (mem[5]) goto code[13];
mem[1] := mem[1] + mem[2];
mem[6] := mem[1];
mem[1] := mem[2];
mem[2] := mem[6];
mem[3] := mem[3] + mem[4];
goto code[5];
output: mem[1];
halt;
\end{minted}

\item Convert to assembly-BDIM
\begin{minted}[linenos,firstnumber=0]{text}
(input,_,_,0)
(read,0,_,1)
(read,1,_,2)
(read,1,_,3)
(read,1,_,4)
(gt,3,0,5)
(if,5,_,13)
(add,1,2,1)
(move,1,_,6)
(move,2,_,1)
(move,6,_,2)
(add,3,4,3)
(goto,_,_,5)
(output,1,_,_)
(halt,_,_,_)
\end{minted}

\item Convert BDIM with strings to BDIM using code.
	\inputminted[linenos,firstnumber=0]{text}{bdim/fib.bdim}
\end{enumerate}

\section{GCD}
\begin{enumerate}
\item \begin{minted}{cpp}
cin >> x >> y;
if (x <= 0) x = -x;
if (y <= 0) y = -y;
while (y != 0) {
	x = x % y;
	z = x;
	x = y;
	y = z;
}
cout << x;
\end{minted}

\item Write branches and loops as goto
\begin{minted}{cpp}
cin >> x >> y;
if (x <= 0) x = -x;
if (y <= 0) y = -y;
L: if (y == 0) goto E;
x = x % y;
z = x;
x = y;
y = z;
goto L;
E: cout << x;
\end{minted}

\item Get rid of temporary values and constants.

\begin{minted}{cpp}
cin >> x;
cin >> y;
zero = 0;
cmp = x > zero;
if (cmp) goto A;
x = zero - x;
A: cmp = y > zero;
if (cmp) goto L;
y = zero - y;
L: cmp = y == zero;
if (cmp) goto E;
x = x % y;
z = x;
x = y;
y = z;
goto L;
E: cout << x;
\end{minted}


\item Write as code/mem state machine.
\begin{minted}[linenos,firstnumber=0]{cpp}
input: mem[1];
input: mem[2];
mem[0] := 0;
mem[4] := mem[1] > mem[0];
if (mem[4]) goto code[6];
mem[1] := mem[0] - mem[1];
mem[4] := mem[2] > mem[0];
if (mem[4]) goto code[9];
mem[2] := mem[0] - mem[2];
mem[4] := mem[2] = mem[0];
if (mem[4]) goto code[16];
mem[1] := mem[1] mod mem[2];
mem[3] := mem[1];
mem[1] := mem[2];
mem[2] := mem[3];
goto code[9];
output: mem[1];
halt;
\end{minted}

\item Convert to assembly-BDIM
\begin{minted}[linenos,firstnumber=0]{text}
(input,_,_,1)
(input,_,_,2)
(read,0,_,0)
(gt,1,0,4)
(if,4,_,6)
(sub,0,1,1)
(gt,2,0,4)
(if,4,_,9)
(sub,0,2,2)
(eq,2,0,4)
(if,4,_,16)
(mod,1,2,1)
(move,1,_,3)
(move,2,_,1)
(move,3,_,2)
(goto,_,_,9)
(output,1,_,_)
(halt,_,_,_)
\end{minted}

\item Convert BDIM with strings to BDIM using code.
	\inputminted[linenos,firstnumber=0]{text}{bdim/gcd.bdim}
\end{enumerate}

\section{Reverse}
\begin{enumerate}
\item \begin{minted}{cpp}
cin >> x;
y = 0;
while (x != 0) {
	y = 10 * y + (x % 10);
	x /= 10;
}
cout << y;
\end{minted}

\item Write branches and loops as goto
\begin{minted}{cpp}
cin >> x;
y = 0;
L: if (x == 0) goto E;
y = 10 * y + (x % 10);
x /= 10;
goto L;
E: cout << y;
\end{minted}

\item Get rid of temporary values and constants.

\begin{minted}{cpp}
zero = 0;
ten = 10;
cin >> x;
y = 0;
L: cmp = x == zero;
if (cmp) goto E;
y = ten * y;
mod = x % ten;
y = y + mod;
x = x / ten;
goto L;
E: cout << y;
\end{minted}

\item Write as code/mem state machine.
\begin{minted}[linenos,firstnumber=0]{cpp}
mem[0] := 0;
mem[1] := 10;
input: mem[2];
mem[3] := 0;
mem[4] := mem[2] = mem[0];
if (mem[4]) goto code[11];
mem[3] := mem[1] * mem[3];
mem[4] := mem[2] % mem[1];
mem[3] := mem[3] + mem[4];
mem[2] := mem[2] / mem[1];
goto code[4];
output: mem[3];
\end{minted}

\item Convert to assembly-BDIM
\begin{minted}[linenos,firstnumber=0]{text}
(read,0,_,0)
(read,10,_,1)
(input,_,_,2)
(read,0,_,3)
(eq,2,0,4)
(if,4,_,11)
(mul,1,3,3)
(mod,2,1,4)
(add,3,4,3)
(div,2,1,2)
(goto,_,_,4)
(output,3,_,_)
(halt,_,_,_)
\end{minted}

\item Convert BDIM with strings to BDIM using code.
	\inputminted[linenos,firstnumber=0]{text}{bdim/reverse.bdim}
\end{enumerate}

\section{Russian Multiplication}
\begin{enumerate}
\item \begin{minted}{cpp}
cin >> x >> y;
res = 0;
while (y != 0) {
	if (y % 2 == 1)
		res += x;
	x = x + x;
	y /= 2;
}
cout << res;
\end{minted}

\item Write branches and loops as goto
\begin{minted}{cpp}
cin >> x >> y;
res = 0;
if (y > 0) goto L;
x = -x;
L: if (y == 0) goto E;
   if (y % 2 == 0) goto SKIP;
   res += x;
SKIP: x = x + x;
   y = y / 2;
   goto L;
E: cout << res;
\end{minted}

\item Get rid of temporary values and constants.

\begin{minted}{cpp}
zero = 0;
two = 2;
cin >> x;
cin >> y;
res = 0;
cmp = y > zero;
if (cmp) goto L;
x = zero - x;
y = zero - y;
L: cmp = y == zero;
if (cmp) goto E;
cmp = y % two;
cmp = cmp == zero;
if (cmp) goto SKIP;
res = res + x;
SKIP: x = x + x;
y = y / two;
goto L;
E: cout << res;
\end{minted}

\item Write as code/mem state machine.
\begin{minted}[linenos,firstnumber=0]{cpp}
mem[0] := 0;
mem[1] := 2;
input: mem[2];
input: mem[3];
mem[4] := 0;
mem[5] := mem[3] > mem[0];
if (mem[5]) goto code[9];
mem[2] := mem[0] - mem[2];
mem[3] := mem[0] - mem[3];
L: mem[5] := mem[3] = mem[0];
if (mem[5]) goto code[18];
mem[5] := mem[3] % mem[1];
mem[5] := mem[5] = mem[0];
if (mem[5]) goto code[15];
mem[4] := mem[4] + mem[2];
SKIP: mem[2] := mem[2] + mem[2];
mem[3] := mem[3] / mem[1];
goto code[9];
E: output: mem[4];
halt;
\end{minted}

\item Convert to assembly-BDIM
\begin{minted}[linenos,firstnumber=0]{text}
(read,0,_,0)
(read,2,_,1)
(input,_,_,2)
(input,_,_,3)
(read,0,_,4)
(gt,3,0,5)
(if,5,_,9)
(sub,0,2,2)
(sub,0,3,3)
(eq,3,0,5)
(if,5,_,18)
(mod,3,1,5)
(eq,5,0,5)
(if,5,_,15)
(add,4,2,4)
(add,2,2,2)
(div,3,1,3)
(goto,_,_,9)
(output,4,_,_)
(halt,_,_,_)
\end{minted}

\item Convert BDIM with strings to BDIM using code.
	\inputminted[linenos,firstnumber=0]{text}{bdim/russian.bdim}
\end{enumerate}
\end{document}
