\documentclass[a4paper]{scrartcl}

\usepackage{minted}
\usepackage{multicol}

\usepackage{todonotes}

\usepackage{tikz}
\usetikzlibrary{automata, positioning, arrows}
\tikzset{
	->, % makes the edges directed
	>=stealth', % makes the arrow heads bold
	node distance=3cm, % specifies the minimum distance between two nodes. Change if necessary.
	every state/.style={thick, fill=gray!10}, % sets the properties for each ’state’ node
	initial text=$ $, % sets the text that appears on the start arrow
}

\usepackage{amsthm, amssymb, amsmath}
\usepackage{bm}

\newtheorem{theorem}{Theorem}
\newtheorem*{theorem*}{Theorem}
\newtheorem{lemma}{Lemma}
\newtheorem*{lemma*}{Lemma}

\newtheorem{claim}{Claim}
\newtheorem*{claim*}{Claim}

\theoremstyle{definition}
\newtheorem{definition}{Definition}
\newtheorem*{definition*}{Definition}

\newcommand{\eps}{\varepsilon}
\newcommand{\card}[1]{\left\lvert #1 \right\rvert}

\newcommand{\NN}{\mathbb N}

\title{
	Programming Languages: Lecture 17\\
	Synthesized and Inherited Attributes
}
\author{Rishabh Dhiman}
\date{10 February 2022}

\begin{document}
\maketitle

\section{Forms of SDDs}
In an SDD, each grammar production rule $X \to \alpha$ associated with it a set of semantic rules of the form $b = f(a_1, \dots, a_k)$ where $a_1, \dots, a_k$ are attributes belonging to $X$ and/or the grammar symbols of $\alpha$.

\begin{definition}
	Given a productions $X \to \alpha$, an attributed $a$ is
	\begin{itemize}
		\item[\textbf{synthesized:}] synthesized attribute of $X$ (denoted by $X.a$) or,
		\item[\textbf{inherited:}] inherited attribute of one of the grammar symbols of $\alpha$ (denoted by $B.a$ if $a$ is an attribute of $B$)
	\end{itemize}
\end{definition}

\section{Attribute Grammars}
An attribute grammar is an SDD in which the functions in semantic rules can have no side-effects.

The attribute grammar also specifies how the attributes are propagated through the grammar, by using graph dependency between the productions.

In general different occurrences of the same non-terminal symbol in each production will be distinguished by appropriate subscripts when defining the semantic rules associated with the rule.

\section{Attributes: Basic Assumptions}
\begin{itemize}
	\item Terminal symbols are assumed to have only synthesized attributes. Their
attributes are all supplied by the lexical analyser during scanning.
\item The start symbol of the grammar can have only synthesized attributes.
	\item In the case of LR parsing with its special start symbol, the start symbol
cannot have any inherited attributes.
\end{itemize}
\end{document}
