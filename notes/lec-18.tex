\documentclass[a4paper]{scrartcl}

\usepackage[T1]{fontenc}

\usepackage{minted}
\usepackage{multicol}

\usepackage{todonotes}

\usepackage{tikz}
\usetikzlibrary{automata, positioning, arrows}
\tikzset{
	->, % makes the edges directed
	>=stealth', % makes the arrow heads bold
	node distance=3cm, % specifies the minimum distance between two nodes. Change if necessary.
	every state/.style={thick, fill=gray!10}, % sets the properties for each ’state’ node
	initial text=$ $, % sets the text that appears on the start arrow
}

\usepackage{amsthm, amssymb, amsmath}
\usepackage{bm}

\newtheorem{theorem}{Theorem}
\newtheorem*{theorem*}{Theorem}
\newtheorem{lemma}{Lemma}
\newtheorem*{lemma*}{Lemma}

\newtheorem{claim}{Claim}
\newtheorem*{claim*}{Claim}

\theoremstyle{definition}
\newtheorem{definition}{Definition}
\newtheorem*{definition*}{Definition}

\newcommand{\eps}{\varepsilon}
\newcommand{\card}[1]{\left\lvert #1 \right\rvert}

\newcommand{\NN}{\mathbb N}

\title{
	Programming Languages: Lecture 18\\
	Symbol Tables and Immediate Representation
}
\author{Rishabh Dhiman}
\date{11 February 2022}

\begin{document}
\maketitle

\section{Symbol Table}
\begin{itemize}
	\item The store house of context-sensitive and run-time information about every
identifier in the source program.
	\item All accesses relating to an identifier require to first find the attributes of the
identifier from the symbol table
	\item Usually organized as a hash table - provides fast access.
	\item Compiler-generated temporaries may also be stored in the symbol table
\end{itemize}

Attributes stored in a symbol table for each identifier,
\begin{itemize}
	\item type
	\item size
	\item scope/visibility information
	\item base address
	\item addresses to location of auxiliary symbol tables
	\item address of the location containing the string which actually names the identifier and its length in the string pool
\end{itemize}\todo{go back and watch or read this portion again}

\section{Intermediate Representation}
IR are important for reasons of portability, ie, platform independence.

\begin{itemize}
	\item No commitment to word boundaries or byte boundaries.
	\item No commitment to representation of,
		\begin{itemize}
			\item int vs. float
			\item float vs. double,
			\item packed vs. unpacked,
			\item strings -- where and how?
		\end{itemize}
\end{itemize}
\end{document}
