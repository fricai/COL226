\documentclass[a4paper]{scrartcl}

\usepackage[T1]{fontenc}

\usepackage{minted}

\usepackage{syntax}

\usepackage{tikz}
\usetikzlibrary{automata, positioning, arrows}
\tikzset{
	->, % makes the edges directed
	>=stealth', % makes the arrow heads bold
	node distance=3cm, % specifies the minimum distance between two nodes. Change if necessary.
	every state/.style={thick, fill=gray!10}, % sets the properties for each ’state’ node
	initial text=$ $, % sets the text that appears on the start arrow
}

\usepackage{amsthm, amssymb, amsmath}
\usepackage{bm}

\newtheorem{theorem}{Theorem}
\newtheorem*{theorem*}{Theorem}
\newtheorem{lemma}{Lemma}
\newtheorem*{lemma*}{Lemma}

\newtheorem*{example}{Example}
\newtheorem*{exercise}{Exercise}
\newtheorem*{notation}{Notation}

\newtheorem{claim}{Claim}
\newtheorem*{claim*}{Claim}

\theoremstyle{definition}
\newtheorem{definition}{Definition}
\newtheorem*{definition*}{Definition}

\newcommand{\eps}{\varepsilon}
\newcommand{\card}[1]{\left\lvert #1 \right\rvert}

\newcommand{\NN}{\mathbb N}

\title{
	Programming Languages: Lecture 10\\
	Syntax Specifications of Programming Languages
}
\author{Rishabh Dhiman}
\date{25 January 2022}

\begin{document}
\maketitle

\section{Extended Backus-Naur Form}
BNF is a CFG specification. Extended comes from its use of regex operators like Kleene Closure and optional specifier.

EBNF specification is a collection of rules which defines the CFG of a language.

PLs in general are Turing-complete. However, CFG cannot specify a Turing machine. So it tends to encode context-sensitive features too.

The syntax analysis of CSGs is very complex.
However, we want parsing to be quick, low-order polynomial time, ideally linear.

Parsing CFGs can be made quick, the language \emph{specification} gives rules for parsing, along with this there are language manual which explains the semantics of these languages. The \emph{semantics} contain context-sensitive features.

The EBNF rules are as follows,
\begin{enumerate}
	\item[Start Symbol.] The very first rule gives the productions of the start symbol of the grammar.
	\item[Non-terminals.] Uses English words or phrases to denote non-terminal symbols. The words or phrases are chosen to be suggestive of the nature or meaning of the constructors.
	\item[Metasymbols.]
		\begin{itemize}
			\item Sequences of constructs enclosed in `\{' and `\}' denote zero or more occurences of the construct (c.f. Kleene Clsoure on regex)
			\item Sequences of constructs enclosed in `[' and `]' are optional, ie, there only be zero or one occurence of the sequence
			\item Constructs are enclosed in `(' and `)' to group them together.
			\item `\textbar' separates alternatives.
			\item `::=' defines the productions of each non-terminal symbol.
			\item `.' terminates the possibly many rewrite rules for a non-terminal.
		\end{itemize}
\end{enumerate}
The same ASCII alphabet is used as both alphabet and operator.

\subsection{Balanced Paranthesis}%
Consider the CFG $BP_3$ given by
\[S \to \eps \mid (S)S \mid [S]S \mid \{S\}S.\]

In EBNF, it will be given by
\setlength{\grammarindent}{10em}
\begin{grammar}
	<BracketSeq> ::= \{<Bracket>\}

	<Bracket> ::= <LeftParen> <BracketSeq> <RightParen>
	\alt <LeftSqbracket> <BracketSeq> <RightSqbracket>
	\alt <LeftBrace> <BracketSeq> <RightBrace>

	<LeftParen> ::= `('

	<RightParen> ::= `)'

	<LeftSqbracket> ::= `['

	<RightSqbracket> ::= `]'

	<RightBrace> ::= `\{'

	<LeftBrace> ::= `\{'
\end{grammar}
You can sepcify EBNF in EBNF, see the hypernotes.

In the olden days we used to enclose EBNF symbols in <>, it's called ASCII-EBNF.

The EBNF of prolog can be fit in just two slides, again see hypernotes.

\section{Regex as a Language}
Given a nonempty finite alphabet A,
\begin{itemize}
	\item every regular language over $A$ is also context-free
	\item the set of regular expression over $A$ is not regular but is context-free
\end{itemize}
This is because of arbitrary nesting and they have associativity and precedence rules.

Both the alphabet of regex and operators of REGEXP(A) happen to be the same.

// There's a tangent about lexical generators here. Go over it when you have the time.
\end{document}
