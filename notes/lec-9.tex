\documentclass[a4paper]{scrartcl}

\usepackage{minted}

\usepackage{tikz}
\usetikzlibrary{automata, positioning, arrows}
\tikzset{
	->, % makes the edges directed
	>=stealth', % makes the arrow heads bold
	node distance=3cm, % specifies the minimum distance between two nodes. Change if necessary.
	every state/.style={thick, fill=gray!10}, % sets the properties for each ’state’ node
	initial text=$ $, % sets the text that appears on the start arrow
}

\usepackage{amsthm, amssymb, amsmath}
\usepackage{bm}

\newtheorem{theorem}{Theorem}
\newtheorem*{theorem*}{Theorem}
\newtheorem{lemma}{Lemma}
\newtheorem*{lemma*}{Lemma}

\newtheorem*{example}{Example}
\newtheorem*{exercise}{Exercise}
\newtheorem*{notation}{Notation}

\newtheorem{claim}{Claim}
\newtheorem*{claim*}{Claim}

\theoremstyle{definition}
\newtheorem{definition}{Definition}
\newtheorem*{definition*}{Definition}

\newcommand{\eps}{\varepsilon}
\newcommand{\card}[1]{\left\lvert #1 \right\rvert}

\newcommand{\NN}{\mathbb N}

\title{
	Programming Languages: Lecture 9\\
	Grammars: Context-Free and Context-Sensitive
}
\author{Rishabh Dhiman}
\date{21 January 2022}

\begin{document}
\maketitle

\section{Grammar}
\begin{definition}[Grammar]
A grammar $G = \langle N, T, P, S\rangle$ consists of
\begin{itemize}
	\item a set $N$ of \emph{nonterminal} symbols.
	\item a \emph{start} symbol $S$.
	\item a set $T$ of \emph{terminal} symbols.
	\item a set $P$ of \emph{productions} or \emph{rewrite rules} where each rule is of the form $\alpha \to \beta$ for $\alpha, \beta \in (N \cup T)^*$.
\end{itemize}
\end{definition}

\begin{definition}
	Given a grammar $G = \langle N, T, P, S\rangle$, any $\alpha \in (N \cup T)^*$ is called a \emph{sentential form}. Any $x \in T^*$ is called a \emph{sentence}.
\end{definition}

\begin{definition}[Context-Free Grammar]
	A grammar $G = \langle N, T, P, S\rangle$ is \emph{context-free} if all production rules are of the form $\alpha \to \beta$ where $\beta$ is a sentential form and $\alpha \in N$.
\end{definition}

\begin{example}[CFG] $\{a^n b^n \mid n > 0\}$ is generated by the grammar,
	\begin{align*}
		S &\to ab,\\
		S &\to aSb.
	\end{align*}
\end{example}

\begin{example}[Context-Sensitive Grammar] $\{a^n b^n c^n \mid n > 0\}$ is generated by the grammar,
	\begin{align*}
		S &\to abc,\\
		S &\to aSBc,\\
		bB &\to bb,\\
		cB &\to Bc.
	\end{align*}
\end{example}

A CFG can be viewed as a CFG where all \emph{contexts} are empty. From what I gather, context is the information around a particular symbol.

\begin{notation}
	We say that $S \stackrel{*}{\Rightarrow} \alpha$ if there's a series of productions such that we can generate $\alpha$ from $S$.
\end{notation}

\begin{definition}
	We see that a grammar is left or right linear when,
	\begin{itemize}
		\item Left-Linear: $X \to a$ or $X \to Y a$.
		\item Right-Linear: $X \to a$ or $X \to a Y$.
	\end{itemize}
\end{definition}

\begin{definition}[Regular Grammar]
	All the productions are either solely left-linear or solely right-linear.
\end{definition}
\begin{theorem}
	Every regular grammar can be generated using right-linear grammar.
\end{theorem}
I think you can encode the non-terminal symbols as states of the DFA.

\section{Derivations}
\begin{definition}[Leftmost Derivation]
	Choose the leftmost non-terminal symbol and apply some production rule.
\end{definition}

Yield of a derivation tree is the string derived.

\section{Ambiguity}
\begin{definition}[Ambiguous Grammar]
	A grammar is said to be \emph{ambiguous} if two sentences can have different derivation trees.
\end{definition}
// Question, does a left-most derivation give rise to a unique derivation tree? It does! (Come up with a tree, thinking of DFS of the tree should help I think)

\begin{definition}[Ambiguous Language]
	A language is said to be \emph{ambiguous} if there's no unambiguous grammar which generates it.
\end{definition}

\section{Removing Ambiguity}
The three common ways adopted to get rid of ambiguity of grammar is,
\begin{itemize}
	\item Introducing bracketing notation
	\item Introducing precedence or associativity rules
	\item Changing the grammar of the language.
\end{itemize}

\begin{exercise}
	Prove/disprove that $S \to \eps \mid a S b S$ generates the same language as $S \to \eps \mid a S b$ and that it generates the same grammar as $S \to SS \mid a S b \mid \eps$.
\end{exercise}

\end{document}
