\documentclass[a4paper]{scrartcl}
\usepackage{csquotes}

\title{Programming Languages: Lecture 2}
\author{Rishabh Dhiman}
\date{6 January 2022}

\begin{document}
\maketitle

Types of languages:
\begin{itemize}
	\item Imperative -- State-based and computation relies on changes of state

		Object-Oriented variant, bundles state and functions on individual members of a class
	\item Functional -- Based on notions of mathematical functions, state plays minor role
	\item Declarative -- Based on logical relations and axioms drawn from logic and mathematics
\end{itemize}

Abstraction levels:
\begin{itemize}
	\item Machine Language -- hides nothing, write in bitsequences
	\item Assembly Language -- hides memory usage related to I/O, exposes the underlying architecture
	\item High-Level Imperative Language -- hides underlying architecture and the structure of memory but exposes individual memory locations through (imperative) variables
	\item Functional Programming -- hides memory and architecture, entirely managing these functions automatically
	\item Declarative Language -- hides everything including algorithmic strategies
\end{itemize}
\end{document}
