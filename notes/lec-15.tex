\documentclass[a4paper]{scrartcl}

\usepackage{minted}
\usepackage{multicol}

\usepackage{tikz}
\usetikzlibrary{automata, positioning, arrows}
\tikzset{
	->, % makes the edges directed
	>=stealth', % makes the arrow heads bold
	node distance=3cm, % specifies the minimum distance between two nodes. Change if necessary.
	every state/.style={thick, fill=gray!10}, % sets the properties for each ’state’ node
	initial text=$ $, % sets the text that appears on the start arrow
}

\usepackage{amsthm, amssymb, amsmath}
\usepackage{bm}

\newtheorem{theorem}{Theorem}
\newtheorem*{theorem*}{Theorem}
\newtheorem{lemma}{Lemma}
\newtheorem*{lemma*}{Lemma}

\newtheorem{claim}{Claim}
\newtheorem*{claim*}{Claim}

\theoremstyle{definition}
\newtheorem{definition}{Definition}
\newtheorem*{definition*}{Definition}

\newcommand{\eps}{\varepsilon}
\newcommand{\card}[1]{\left\lvert #1 \right\rvert}

\newcommand{\NN}{\mathbb N}

\title{
	Programming Languages: Lecture 15\\
	I have no clue
}
\author{Rishabh Dhiman}
\date{4 February 2022}

\begin{document}
\maketitle

\section{Static}

Before execution (runtime)
\begin{itemize}
	\item compile time
	\item linking time
	\item loading time
\end{itemize}

Context-sensitive analysis (static semantics)
Type, size, requirements, scopes, etc.

Static memory allocation.

FORTRAN 4:
All data sizes are known at compile-time, there is no nesting of subroutine or recursion.

Memory allocation for all sub-routines and the main routine allocated at compile-time. Memory is divided up into essentialy disjoint units so that there is no overlap.

\subsection{Static allocation}
\begin{itemize}
	\item[+] faster execution because there are no overheads due to fixed predefined address for all data
	\item[-] inflexible
	\item[-] does not allow for recursion or nested scoping
	\item[-] wasterful over time because a lot of memory is simply lying idle and is not used.
\end{itemize}
Lot of weather-prediction programs currently used use FORTRAN because of \textit{speed}.

\section{Dynamic}

At execution-time (runtime), meaning of program.

\subsection{Dynamic Memory Allocation}
Almost all other languagws which support
\begin{itemize}
	\item nested scoping
	\item recursion
	\item dynamic data-structures
	\item all OO, functional and declarative languages
\end{itemize}

Most algol languages, starting from ALGOL60.

\begin{itemize}
	\item[+] Flexible allocation at runtime
	\item[+] Memory may be more optimally used on demand.
	\item[-] Impossible to ascertain how much memory is actually required at any time, because it depends on the calling-chain at any instant of program execution.
	\item[-] Impossible to ascertain how many recursive activations of a procdure or function may be required at any instant of time during execution.
	\item[-] Impossible to ascertain how many different objects of various sizes need to be present simulataneously during runtime.
\end{itemize}

\section{Run-Time Environment}
Memory for running a program is divided up as follows,

\begin{itemize}
	\item[\textbf{Code segment.}] This is where the \emph{object} code of the program resides.
	\item[\textbf{Runtime stack.}] Required in a \emph{dynamic} memory management technique. Especially required in languages which support \emph{recursion}.
		All data whose sizes can be determined \emph{statically} before loading is stored in an appropriate \emph{stack-frame} (activation road).
	\item[\textbf{Heap.}] All data whose sizes are not determined statically and all data that is there is genreated at run time.
\end{itemize}

Consider the following program,
\begin{minted}{text}
// Main Program
// global vars
{
	// Procedure P2
	// Locals of P2
	{
		// Procedure P21
		// Locals of P21
		// Body of P21
	}
	// Body of P2
}

{
	// Procedure P1
	// Locals of P1
	// Body of P1
	
	// Call P2
}
// Main body
\end{minted}
with the following calling chain,
\[\text{Main} \to P1 \to P2 \to P21 \to P21.\]

At the end of P2, all global variables must be present, P1 variables should still exist but shouldn't be accessible. P21 made a recursive call to P21, the same variables should be allocated again however, these should be allocated at a different location as before.

The memory is divided up as follows,

\begin{tikzpicture}

\end{tikzpicture}

When stack overflows heap, segmentation faults occur.

The stackfram or activation record looks as follows,
\begin{minted}{text}
Return address to last of P21.
	Dynamic link to last P2.
Locals of P21.
Static link to last P2.
Formal par of P21.

-----------
Return address to last of P2.
	Dynamic link to last P2.
Locals of P21.
Static link to last P21.
Formal par of P21.

-----------
Return address to last of P1.
	Dynamic link to last P1.
Locals of P2.
Static link to main.
Formal par of P2.

-----------
Return address ot Main
	Dynamic Link to main
Locals of P1
Static Link to Main
Formal par of P1

--------------
Globals
\end{minted}
Address are all relative to the base address of an activation.

The static link specifies the non-local referencing environment. It points to its `parent' in the scoping structure and tells what variables other than the local variables it's accessing and to get all these non-local variables we have to go through all these static links to the root, the number of transitions specify the nesting depth, 0 transitions -- local variables, etc.

\end{document}
