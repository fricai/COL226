\documentclass[a4paper]{scrartcl}

\usepackage{minted}

\usepackage{amsthm, amssymb, amsmath}
\usepackage{bm}

\newtheorem{theorem}{Theorem}
\newtheorem*{theorem*}{Theorem}
\newtheorem{lemma}{Lemma}
\newtheorem*{lemma*}{Lemma}

\newtheorem{claim}{Claim}
\newtheorem*{claim*}{Claim}

\theoremstyle{definition}
\newtheorem{definition}{Definition}
\newtheorem*{definition*}{Definition}

\newcommand{\eps}{\varepsilon}

\title{
	Programming Languages: Lecture 6\\
	Lexical Analysis
}
\author{Rishabh Dhiman}
\date{15 January 2022}

\begin{document}
\maketitle

\section{Lexical Analysis}
A source program consists of a stream of characters.

Given a stream of characters that make up a source program the compiler must first break up this stream into a sequence of ``lexemes" and other symbols.

Lexemes are often separated by non-lexemes.

Certain sequences of characters are \emph{not} tokens (Eg.: comments)

\subsection{Erroneous Lexemes}
Some lexemes violare all rules of tokens.
\begin{itemize}
	\item 12ab would not be an identifier or a number in most languages, 0x2ab may be hex.
	\item 127.0.1 probably won't be a number but 127.0.0.1 is a valid IP.
\end{itemize}

\subsection{Tokens}
Common examples
\begin{itemize}
	\item Constants
	\item Identitfiers -- Name of variables, constants, procedures, functions, etc.
	\item Keywords/Reserved words -- void, public, main
	\item Operators -- +, *, /
	\item Punctuation -- ,, :, .
	\item Brackets -- (, ), [, ], begin, end, case, esac
\end{itemize}

\subsection{Scanning}
During the scanning phase the compiler/interpreter
\begin{itemize}
	\item takes a stream of characters and identifies tokens from the lexemes
	\item eliminiates comments and redundant whitespace
	\item keeps track of line numbers and columb numbers and passes them as parameters to the other phases to enable error-reporting and handling to the user.
\end{itemize}

\section{Regular Expressions Language}
\begin{itemize}
	\item Any set of strings built up from the symbols of $A$ is called a language. $A^*$ is the set of all finite strings buillt up form $A$.
	\item Each regex is a finite sequence of symbols made up of symbols from the alphabet and other symbols called operators.
	\item A regular expression may be used to describe an \emph{infinite} collection of strings.
\end{itemize}

\section{Language}
Any collection of finite strings is a language.

\section{Simple Language of Regular Expressions}
We consider a simple language of regular expressions. Assume a (finite) alphabet $A$ of symbols. Each regular expression $r$ denotes a set of strings $\mathcal L(r)$. $\mathcal L(r)$ is also called the \emph{language} spexified by the regular expression $r$.
\begin{itemize}
	\item Symbol, for $a \in A$, $\{a\}$ refers to the single element $a$.
	\item Concatenation. $\mathcal{L}(rs) = \mathcal{L}(r)\mathcal{L}(s)$.
	\item Epsilon $\eps$ denotes the language with a single element the \emph{empty} string, ``\;".
		\[L(\eps) = \{\eps\}.\]
	\item Alternation. Given two regex $r, s$; $r \mid s$ is the set of union of the languages specified by $r$ and $s$.
		\[\mathcal{L}(r \mid s) = \mathcal{L}(r) \cup \mathcal{L}(s).\]
	\item Kleene Closure $r^* = r^0 \mid r^1 \mid \cdots$ denotes an infinite union of languages.
		\[\mathcal{L}(r^*) = \bigcup_{n = 0}^{\infty} \mathcal{L}(r^n).\]
\end{itemize}
\end{document}
